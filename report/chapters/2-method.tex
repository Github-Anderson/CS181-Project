% filepath: report/2-method.tex
\section{Methodology}
In this section, we will introduce our implementation about our various agents based on the original rules.

\subsection{State Evaluation}
In search and adversarial search, state evaluation function is extremely important. It provides assessment information about a state and the agent will utilize  it and act accordingly.

In our implementation, our state evaluation function calculate the Euclidean distance from the pawn to every empty or non-player-occupied position in the player's goal area. If there are available goal positions, add the maximum of these distances to a running total.
If there are no available goal positions for that pawn, add 20 from the total as a penalty, since it indicates that there are opponents' remaining pawn in the area and moving towards the area may prevent them from leaving because of crowding pawns. 

After evaluating all pawns: Multiply the total score by -1, so that a smaller distance (i.e., pawns closer to the goal) results in a higher evaluation score. The state evaluation function equation is as follows:

\begin{equation}
\text{val} = - \sum_{p \in P_{\text{self}}} \left\{
\begin{array}{ll}
\max_{g \in G_{\text{empty}}} \text{dist}(p, g), & \text{if } G_{\text{empty}} \neq \emptyset \\
-20, & \text{if } G_{\text{empty}} = \emptyset
\end{array}
\right.
\end{equation}
where the distance is Euclidean one:
\begin{equation}
    \text{dist}(p, g) = \sqrt{(x_p - x_g)^2 + (y_p - y_g)^2}
\end{equation}

\subsection{RandomPlayer}
Every time \texttt{RandomPlayer} takes the turn, it will take a random valid action. 

\subsection{GreedyPlayer}
Every time \texttt{GreedyPlayer} takes the turn, it will take all the valid moves into account, and use the State Evaluation function to assess the corresponding following state as action score. Then \texttt{GreedyPlayer} will always choose the move with the maximum action score.

\subsection{Minimax Agent}
The \texttt{MinimaxPlayer} utilizes the classic Minimax algorithm, a recursive search enhanced with alpha-beta pruning and optional local search technique for efficiency.

\subsubsection{Core Algorithm}
\texttt{MinimaxPlayer} explores the game tree to a predefined depth due to rather large state space, assuming the opponent will always choose moves to minimize the Minimax player’s score. Alpha-beta pruning is employed to eliminate branches of the search tree that won't influence the final decision but significantly speeding up the search. In our experiments, Minimax agents has limited depth of two. 

\subsubsection{Local Search (Optional)}
If enabled, Local Search Heuristic aims to further prune the search space at each node of the Minimax tree and reduce the branching factor. Instead of considering all legal moves, it first evaluates the best possible move for each individual pawn based on the state evaluation score. Only this collection of best individual pawn moves is then considered by the Minimax algorithm. 


\subsection{Monte Carlo Tree Search Agent}
The \texttt{MCTSPlayer} utilizes Monte Carlo Tree Search\cite{DBLP:journals/corr/abs-2103-04931}, a probabilistic search algorithm that balances exploration of new possibilities with exploitation of known good paths.

\subsubsection{Core Process:} 
MCTS iteratively builds a search tree. Each iteration involves four phases as shown in Figure ~\ref{fig:mcts}:
\begin{figure}[h]
    \centering
    \includegraphics[width=1\columnwidth]{figures/mcts.png}
    \caption{Illustration of MCTS Core Process}
    \label{fig:mcts}
\end{figure}

\paragraph{Selection}
Starting from the root (current game state), the algorithm traverses the existing tree by repeatedly choosing child nodes that maximize the Upper Confidence Bound (UCB) criterion. In our implementation, the UCB formula is as follows:
\begin{equation}
    UCB = \frac{Q}{N} + c \sqrt{\frac{\ln N_{\text{parent}}}{N}} + \text{strategy\_score}
\end{equation}
where $Q$ is the cumulative value, $N$ is the visit count, $c$ is the exploration parameter and the $strategy\_score$ function for any move action $a$ is as follows:
\begin{equation}
StrategyScore(a) = 0.2 \times direction + jump
\end{equation}

\begin{equation}
direction=
\begin{cases} 
\frac{(x_e - x_s) + (y_e - y_s)}{2} & \text{if player } P_{\text{child}} \text{ is "RED"} \\
\frac{(x_s - x_e) + (y_s - y_e)}{2} & \text{if player } P_{\text{child}} \text{ is "GREEN"}
\end{cases}
\end{equation}

\begin{equation}
jump=
\begin{cases} 
0.3 & \text{if action } a \text{ is a jump} \\
0 & \text{otherwise}
\end{cases} 
\end{equation}
where $(x_e, y_e)$ stands for the position of the pawn after the move and $(x_s, y_s)$ stands for the one before the move. Note that in our implementation, Red Player's goal area is located at the right-bottom diagon and Green Player's goal area is located at the left-up diagon. The item \textit{direction} and \textit{jump} will encourage the agent to explore actions that move toward the goal area and jump, accordingly. 

\paragraph{Expansion}
If the selection process reaches a leaf node that is not a terminal game state and has untried actions, one new child node is added to the tree, corresponding to an untried action. Actions are prioritized for expansion based on heuristic scores (distance improvement, jump bonus, backward penalty).

\paragraph{Simulation}
From this new node (or a selected leaf if it's terminal), a simulated game (playout) is conducted. Actions during simulation are chosen using a fast, heuristic policy that favors moves improving distance to goal, direction, and jumps, with a small chance of random action selection. The playout continues until a game end-state or a depth limit.

\paragraph{Backpropagation}
The outcome of the simulation is propagated back up the tree from the expanded node to the root, updating the visit counts and value estimates of all traversed nodes. Under most circumstance, the simulation can't reach the end, hence requiring a \textit{Simulation Evaluation} function. We will soon introduce that. 

\subsubsection{Simulation Evaluation}
If a simulation ends due to depth limit rather than game completion, this function provides a heuristic score. In our implementation, we design a comprehensive simulation function heuristic which considers the number of pieces in the goal, the average distance of pieces to the goal, progressive bonuses for achieving stages of goal occupation, and penalties for pieces remaining in the starting area. The comprehensive monte carlo tree search simulation evaluation funciton is as follows for a state:
\begin{equation}
\begin{split}
Score_{\text{eval}} = \text{clamp}(Score_{\text{goal}} + Score_{\text{dist}} +
Bonus_{\text{stage}} + \\ Penalty_{\text{home}}, -1000, 1000)
\end{split}
\end{equation}
\begin{equation}
\text{clamp}(x, a, b) = \max(a, \min(x, b))
\end{equation}
\begin{equation}
Score_{\text{goal}} = 100 \times P_G
\end{equation}
\begin{equation}
Score_{\text{dist}} = 90 \times \left(1 - \frac{D_{\text{norm\_sum}}}{N_P}\right)
\end{equation}
\begin{equation}
Bonus_{\text{stage}} =
\begin{cases}
0 & \text{if } P_G = 0 \\
50 & \text{if } P_G = 1 \\
150 & \text{if } P_G = 2 \\
350 & \text{if } P_G = 3 \\
750 & \text{if } P_G \geq 4 
\end{cases}
\end{equation}
\begin{equation}
Penalty_{\text{home}} = -100 \times \frac{P_H}{N_P}
\end{equation}
where $P_G$ is the number of the player's pieces in the goal area,
$N_P$ is the total number of the player's pieces,
$D_{norm\_sum}$ is the sum of normalized Manhattan distances to the goal center for all pieces not in the goal, and 
$P_H$ is the number of the player's pieces in the home area and not moved. The intention of designing complicated simulation function is to provide a stronger evaluation with more inductive bias which may empirically contribute to the performance of our \texttt{MCTSPlayer}.

\subsubsection{Final Action Selection}
After a set number of simulations or a time limit, the agent chooses the action from the root's children that is most promising, based on a weighted combination of its win ratio, visit count, and a directional score.
\begin{equation}
\begin{split}
    score = 0.4 \times win\_ratio + 0.2 \times visit\_ratio \\ +  0.4 \times direction
\end{split}
\end{equation}

\begin{equation}
win\_ratio = \frac{child.value}{child.visits}
\end{equation}

\begin{equation}
visit\_ratio = \frac{child.visits}{root.visits}
\end{equation}

\begin{equation}
direction =
\begin{cases}
(x_e - start_x) + (y_e - y_s) & \text{if color is RED} \\
(x_s - x_e) + (y_s - y_e) & \text{otherwise}
\end{cases}
\end{equation}
where $(x_e, y_e)$ stands for the position of the pawn after the move and $(x_s, y_s)$ stands for the one before the move. This mechanism is to enhance our inductive bias and force \texttt{MCTSPlayer} to behave more wisely. Note that due to the complicated procedures, \texttt{MCTSPlayer} takes apparently more time in a turn to decide a move. 

\subsection{Q-learning Agent(Failed)}
Unfortunately, we failed to train a Q-learning agent. We tried letting Q-learning agent fight against random/minimax/Q-learning agents in limited episodes, but results in unintelligent behaviors. If setting the episodes larger, the q-state information file will be drastically large. This may caused by the rather large state space. According to our experiment, the file containing trained parameters after 200 episodes in .txt file is 21GB. When loading it into python, the program will crash.  

\subsection{Approximate Q-Learning Agent}
This agent learns to play Halma using Q-learning with linear function approximation. It utilizes various feature functions use linear combination with learnable weights to score a state. In short, the Q-value is approximated as a weighted sum of features: $Q(s,a) = \sum w_i f_i(s,a)$ where the weights $w_i$ can be learned.

\subsubsection{Feature Engineering}
In our implementation, we design a set of handcrafted features, $f_i(s,a)$, which describe the state-action pair. These include: normalized count of pieces in the goal, average distance of pieces to the goal, improvement in distance to goal due to the action, directional score of the action, and binary indicators for jumps, reaching the goal, moving backwards, or leaving the home area. Initial heuristic weights are assigned to these features.

In detail, our implemented feature functions and the final q-state approximation are as follows:
% State features
\begin{equation}
pieces\_in\_goal = \frac{\text{Number of player's pieces in goal area}}{4}
\end{equation}

\begin{equation}
avg\_distance = \frac{1}{4D_{\max}} \sum_{p \notin G} \left| x_p - x_c \right| + \left| y_p - y_c \right|
\end{equation}

% Action features
\begin{equation}
distance\_improvement = \frac{d_{\text{start}} - d_{\text{end}}}{D_{\max}}
\end{equation}

\begin{equation}
direction = 
\begin{cases} 
\frac{(x_{\text{e}} - x_{\text{s}}) + (y_{\text{e}} - y_{\text{s}})}{2B}, & \text{if RED} \\
\frac{(x_{\text{s}} - x_{\text{e}}) + (y_{\text{s}} - y_{\text{e}})}{2B}, & \text{otherwise}
\end{cases}
\end{equation}

\begin{equation}
is\_jump = 
\begin{cases} 
1, & \text{if jump move} \\
0, & \text{otherwise}
\end{cases}
\end{equation}

\begin{equation}
reaches\_goal = 
\begin{cases} 
1, & (x_{\text{e}}, y_{\text{e}}) \in G \\
0, & \text{otherwise}
\end{cases}
\end{equation}

% Additional action features
\begin{equation}
is\_backwards = 
\begin{cases} 
1, & \text{if move is backwards} \\
0, & \text{otherwise}
\end{cases}
\end{equation}

\begin{equation}
leaves\_home = 
\begin{cases} 
1, & (x_{\text{s}}, y_{\text{s}}) \in H \text{ and } (x_{\text{e}}, y_{\text{e}}) \notin H \\
0, & \text{otherwise}
\end{cases}
\end{equation}

\begin{equation}
\begin{split}
Q(s, a) = w_1 \cdot \text{pieces\_in\_goal} + w_2 \cdot  \text{avg\_distance} + 
\\ w_3 \cdot \text{distance\_improvement} + w_4 \cdot \text{direction} + \\ w_5 \cdot \text{is\_jump} + w_6 \cdot \text{reaches\_goal} + 
\\ w_7 \cdot \text{is\_backwards} + w_8 \cdot \text{leaves\_home}
\end{split}
\end{equation}

\subsection{Learning Mechanism}
Weights are updated using the Temporal Difference (TD) error. After taking an action $a$ from state $s$, observing reward $r$ and next state $s'$, the TD error is:
\begin{equation}
    \delta = r + \gamma \max_{a'} Q(s',a') - Q(s,a)
\end{equation}
Each weight $w_i$ is updated by:
\begin{equation}
w_i \leftarrow w_i + \alpha \cdot f_i(s,a)
\end{equation}
where $\alpha$ is the learning rate and $\gamma$ is the discount factor.

In our implementation, we support two approach for learning the weights: \textit{learning while fighting} and \textit{specific training}. For \textit{learning while fighting}, we use an empirically promising initial weights and use $\epsilon$-greedy strategy to update the weights and take actions during the competition. For \textit{specific training}, we have a specific training script of letting the agent to fight against minimax when being sente or gote, and update the weights.

According to our experiments and attempts, \textit{learning while fighting} strategy performs much better than \textit{specific training}. Hence we solely consider the approximate q-learning agent with \textit{learning while fighting}.

\subsection{$\epsilon$-greedy}
The agent balances exploration (trying new actions) and exploitation (choosing the best-known action). With probability $\epsilon$, it explores (choosing a non-backward random move if possible); otherwise, it exploits by selecting the action with the highest current Q-value. The exploration rate $\epsilon$ dynamically adjusts based on game progress and decays over time.
\subsection{Reward Function}
In approximate q-learning, designing reward is also important. The Q-value will be tuned towards the pattern of reward function. We provide a multi-stage and comprehensive reward function of a state and corresponding action for our agent as follows:
\begin{equation}
\begin{split}
& \text{reward} = \mathbf{1}_{\text{win}} \cdot \left(3000 + 500 \times 4 \times pieces\_in\_goal\right)
\\
&+ \mathbf{1}_{goal\_progress > 0} \cdot \left[300 \times 2^{current\_pieces}\right]
\\
&+ 
\begin{cases}
200 \times distance\_improvement, & \text{if } 4 \times pieces\_in\_goal \geq 2 \\
100 \times distance\_improvement, & \text{otherwise}
\end{cases}
\\
&- \mathbf{1}_{avg\_distance^{new} > avg\_distance^{old}} \cdot 300
\\
&+ 
\begin{cases}
0, & \text{if } is\_jump = 1 \text{ and } 4 \times pieces\_in\_goal \geq 3 \\
50, & \text{if } is\_jump = 1 \text{ and } 4 \times pieces\_in\_goal \geq 2 \\
200, & \text{if } is\_jump = 1 \text{ and } 4 \times pieces\_in\_goal < 2 \\
0, & \text{otherwise}
\end{cases}
\\
&-
\begin{cases}
500, & \text{if } is\_backwards = 1 \text{ and } 4 \times pieces\_in\_goal \geq 2 \\
200, & \text{if } is\_backwards = 1 \text{ and } 4 \times pieces\_in\_goal < 2 \\
0, & \text{otherwise}
\end{cases}
\end{split}
\end{equation}
In short, it provides large rewards for winning, scaled bonuses for pieces entering the goal, rewards for distance improvement (scaled by game stage), penalties for moving backward, and dynamic bonuses for jumps (larger in early game). The intention of designing complicated and comprehensive reward function is to make the reward more intuitively consistent with the real game reward pattern. 
% \end{itemize}

\subsection{Neural Approximate Q-Learning Agent (DQN)}
This agent implements a Deep Q-Network (DQN), a more advanced reinforcement learning technique that uses a neural network to approximate the Q-function. The unique components compared to Approximate Q-learning are Q-value network and DQN training paradigm. 
\subsection{Network Pipeline}
The pipeline of Network is as Figure ~\ref{fig:pipeline}. The input of network is Board state (4 channels) and action (4 dimensional). Board state has four channels encoding the board states containing player's pawns, opponent's pawns, player's goal area and opponent's goal area. The action is four dimensional since it has four data: $x_{\text{start}}, x_{\text{end}}, y_{\text{start}}, y_{\text{end}}$. The board vector will go through three convolutional layers (Conv2d)\cite{DBLP:journals/corr/OSheaN15} and action vector will go through a fully connected layer. These two processed vector will be concatenated and fed into three fully connected layers and eventually output Q-value. This is a regression model. 
\begin{figure}[h]
    \centering
    \includegraphics[width=1\columnwidth]{figures/sente.png}
    \caption{Training figure of sente side}
    \label{fig:sente}
\end{figure}
\begin{figure}[h]
    \centering
    \includegraphics[width=1\columnwidth]{figures/gote.png}
    \caption{Training figure of gote side}
    \label{fig:gote}
\end{figure}
\begin{algorithm}[H]
\caption{Deep Q-learning with Experience Replay}
\label{alg:dqn}
\begin{algorithmic}[1]
\State Initialize replay memory $\mathcal{D}$ to capacity $N$
\State Initialize action-value function $Q$ with random weights
\For{episode $= 1, M$}
    \State Initialize sequence $s_1 = \{x_1\}$ and preprocessed sequenced $\phi_1 = \phi(s_1)$
    \For{$t = 1, T$}
        \State With probability $\epsilon$ select a random action $a_t$
        \State otherwise select $a_t = \max_{a} Q^*(\phi(s_t), a; \theta)$
        \State Execute action $a_t$ in emulator and observe reward $r_t$ and image $x_{t+1}$
        \State Set $s_{t+1} = s_t, a_t, x_{t+1}$ and preprocess $\phi_{t+1} = \phi(s_{t+1})$
        \State Store transition $(\phi_t, a_t, r_t, \phi_{t+1})$ in $\mathcal{D}$
        \State Sample random minibatch of transitions $(\phi_j, a_j, r_j, \phi_{j+1})$ from $\mathcal{D}$
        \State Set $temp = \gamma \max_{a'} Q(\phi_{j+1}, a'; \theta)$
        \State Set $y_j = \left\{
            \begin{array}{ll}
                r_j & \text{for terminal } \phi_{j+1} \\
                r_j + temp & \text{for non-terminal } \phi_{j+1}
            \end{array}
        \right.$
        \State Perform a gradient descent step on $(y_j - Q(\phi_j, a_j; \theta))^2$
    \EndFor
\EndFor
\end{algorithmic}
\end{algorithm}
\begin{figure}[h]
    \centering
    \includegraphics[width=1\columnwidth]{figures/pipeline.png}
    \caption{Pipeline of Network in DQN}
    \label{fig:pipeline}
\end{figure}
\subsection{DQN training paradigm}
The overall pseudocode about DQN\cite{DBLP:journals/corr/MnihKSGAWR13} training is presented in Pseudocode Algorithm ~\ref{alg:dqn}.

\subsubsection{Experience Replay}
Transitions (state, action, reward, next state, done flag) are stored in a replay memory. During training, mini-batches are randomly sampled from this memory to update the network. This breaks correlations between consecutive samples and improves learning stability.

\subsubsection{Target Network}
A separate "target" neural network, with the same architecture as the main Q-network, is used to generate the target Q-values for the TD error calculation ($R + \gamma \max_{a'} Q_{\text{target}}(s',a')$). The weights of the target network are periodically copied from the main Q-network, providing a more stable learning target.

\subsubsection{Learning Mechanism} 
The main Q-network is trained by minimizing the Mean Squared Error (MSE) between its predicted Q-values and the target Q-values computed using the target network and observed rewards. The Adam optimizer is used.
The related training figures in our experiments when training on the sente side and gote side are as Figure~\ref{fig:sente} and~\ref{fig:gote}.